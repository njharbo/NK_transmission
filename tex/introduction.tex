\section{Introduction}
\lb{introduction}

[NK under CM studied as approx.] New Keynesian models are often studied under the assumption of complete market. This assumption seems hard to square with the empirical evidence, but the relevance of the models can still be justified insofar that they serve useful approximations of a world with incomplete markets or that the effects of heterogeneity can be studied additively to the effects under the complete market assumption. 

[Validity hinges on supply side] In this paper we show that the construction of the supply side is key for the validity of this presumption. For this purpose we consider a simple heterogenous agent models with two classes of agents: workers who supply labor and only receive labor market income, and capitalists who receive firm profits and do not supply any labor. We consider the effect of monetary policy under two different assumptions about the sources of nominal rigidities in the model. When nominal ridigities derives only from pricing frictions in the goods market, the is no effect on output from monetary policy. In contracts, monetary policy produces response in output when there is a sufficient degree of rigidity in wage setting. 

[Intuition] The intuition behind these result lies on the model's supply side. This is often ignored in the exposition of New Keynesian models with complete markets, but we illustrate that it lies at the heart of the transmission mechanism. In the model with nominal price ridigities workers operate on their intratemporal optimality condition, why labor supply is governed by substitution and income effects. As the model exhibits balanced growth path preferences, these two will cancel out in the absence of a source of non-labor income. Consequently, wage changes will not be able to affect labor supply, why output becomes invariant to monetary policy. The mechanism of the supply side is changed once we also introduce wage ridigities. Here workers might be unable to reset their wages in a given period, why labor supply no longer is given by income and substitution effects. Instead workers supply the labor demanded by firms, and consequently the absence of profits in the budget set of the worker does not make output invariant to monetary policy.

[Highlights counter-factual in the textbook model] Our result also high-lights a counter-factual, but unappreciated, feature of the transmission mechanism of the New-Keynesian textbook model \citep{Gali2009}. Usually, the demand side of the transmission mechanism in this model is emphasized.\footnote{The standard intuition is that subject to a surprise increase in the nominal interest rate, the real interest rate increase due to sticky prices, and aggregate demand falls due to intertemporal substitution. Since, in the absence of capital, consumption equals output in equilibrium, output falls.} However, as in any general equilibrium model, the movement in output must not only be consistent with consumption demand, but also with the supply of factor inputs, which in the textbook model solely consist of labor. To make labor supply move in response to a monetary shock, the size and distribution of profits play a key role. First, the mere existence of profits in the budget set of the representative agent decreases the relative income effect, and thus allows wages to affect labor supply even in the presence of balance growth path preferences. This effect is sizable since profits make up 45 \% of steady state output under a standard calibration of the textbook model. Second, owing to sticky prices average markups and profits respond countercyclically to monetary policy. This gives rise to a countercyclical income effect on labor supply, which helps to increase labor supply in response to a decrease in the nominal interest rate and \emph{vice versa}.

[Literature on NK models with agent heterogeneity] Our analysis connects to the recent literature analyzing the effect of household income and wealth heterogeneity in New-Keynesian environments, since our worker-capitalist model can be seen as highly stylized model of the same type of heterogeneity. \citet{Auclert2015} studies amplification of monetary policy through redistribution and heterogeneity in marginal propensities to consume. \citet{McKay2015} studies how the precautionary savings motive affects the responsiveness to forward-guiding monetary policy. \citet{McKay2013} and \citet{Ravn2012a} studies richer model to uncover the effect of job uncertainty and automatic stabilizers. 

[Literature using the text-book model]
Secondly, the paper relates to the literature using the 3-equation model for analysis of the business cycle and business cycle policy. To name a few, \citet{Lorenzoni2009} studies the effect of news shocks. \citet{Christiano2011} analyzes government spending multipliers. \citet{Werning2012} characterize optimal fiscal and monetary policy at the zero lower bound. 

[Organization.] The remainder of the paper is organized as follows. In Section \ref{sec:model} we set up our heterogeneous agent pertubation of the standard 3-equation model and relate it to the textbook model. In Section \ref{sec:results} we use these two models to analyze monetary and TFP shocks. Section \ref{sec:bond_trade} discusses how our results extend to a heterogenous agent model with financial trade between the two classes of agents, while Section \ref{sec:wage_rigidity} discusses how the existence of wage stickyness affect our results. Section XX concludes. 