\begin{abstract}
We investigate to what extent household heterogeneity affects the transmission mechanism of monetary policy in the textbook version of the New Keynesian model. For this purpose, we extend the textbook model to contain two classes of households: workers who earn wages, and capitalists who earn profits. We show that the source of nominal rigidities matters greatly for the monetary transmission mechanism in this framework. In the model with only price rigidities, there is no effect on output from monetary policy. This result sharply contrasts with the corresponding representative agent model, in which the large steady state size and countercyclical response of profits explain the procyclical response of labor supply and output. In the model with frictional labor markets, the response of employment and output becomes essentially demand-determined, and the worker-capitalist model behaves closely to the representative agent model. 
\end{abstract}