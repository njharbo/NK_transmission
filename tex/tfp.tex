\section{TFP shocks}
\label{sec:tfp}

Our paper has centered around the transmission mechanism of the standard model to monetary policy shocks. However, as we showed in Section \ref{sec:results}, without the redistribution of profits to the representative household, labor supply is unresponsive to any movement in real wages. This is because without any non-labor income, KPR preferences implies that the income and substitution effect cancel with respect to changes in the wage level. 

We follow this insight to investigate the transmission mechanism from TFP shocks to employment. In contrast to standard real business cycle models, the three-equation new-Keynesian model generates a fall in employment in response to a TFP shock. This impulse-response has been documented in US data \citep{Gali1999,Gali2004,Francis2005,Basu2006}, and has been interpreted as evidence in favor of the new-Keynesian model. 

We proceed in the same fashion as in Section \ref{sec:results}. We consider how the standard and the worker-capitalist models respond to the same innovation in TFP. We assume that innovations in TFP follows the process
\begin{eqnarray}
a_t &=& \rho_{a} a_{t-1}+\epsilon_{a t} \nonumber
\end{eqnarray}
with $\rho_{\nu}=0.9$. We feed a positive $1$ percentage point shock to the models. The responses are plotted in Figure \ref{fig_tfp_std_nt}. Note that in the figure, we do not plot deviations from the flex price equilibrium, but rather deviations from steady state.

As seen in Figure \ref{fig_tfp_std_nt}, the standard model generates a fall in hours worked to the positive innovation in TFP. On the demand side, this response is often interpreted as an effect of the (suboptimal) Taylor rule. In response to the rise in TFP, the unit marginal cost of production falls, which generates a fall in inflation. The central bank then stimulates the economy through the Taylor rule, but not sufficiently to close the real interest rate gap. Although output rises, we then see a fall in the output gap and in hours worked. 

Turning to the supply side, the increase in real wages has a positive effect on labor supply, ceteris paribus. This effect arises due to profit income being large in steady state, so that the relative income effect of wages is depressed. The fall in hours must therefore be a consequence of the large increase in profits, which brings a positive income effect. 

Again, this transmission mechanism does not survive the worker-capitalist model. In this model, workers do not experience the positive income effect from the rise in profits. Workers only experience the rise in wages, but with KPR preferences, this does not affect labor supply. Hence, it is not only the design of monetary policy that is crucial for the response in hours in the standard model, but also the income effect from the large rise in profits.
