\begin{abstract}
The success of the textbook three-equation New Keynesian textbook model stems from its capability to match the response of output to innovations in monetary policy. We reexamine the transmission channel of the textbook model and show that this successful result depends on the assumption that firm profits are redistributed to working households. Because these profits are large and respond countercyclically, labor supply increases after an expansionary monetary policy shock. We contrast the textbook model to a simple modification where profits are consumed by non-working capitalists. This modification renders output unresponsive to monetary policy. Expanding the textbook model with sufficiently rigid labor markets, the response of profits is no longer instrumental for the response of employment and output.
\end{abstract}