\section{Introduction}
\lb{introduction}
The 3-equation New-Keynesian textbook model is a standard tool for investigating business cycle movements and policies. While having the virtue of being minimal in assumptions, it can still qualitatively match the aggregate responses to various shocks documented in the data. Most notably, it can generate a negative response of output to positive shocks in the nominal interest rate. 

We reexamine the transmission mechanism of this model. The standard intuition is that subject to a surprise increase in the nominal interest rate, the real interest rate increase due to sticky prices, and aggregate demand falls due to intertemporal substitution. Since, in the absence of capital, consumption equals output in equilibrium, output falls. However, as in any general equilibrium model, the movement in output must not only be consistent with consumption demand, but also with the supply of factor inputs, which in the 3-equation model solely consist of labor. The supply part of the model is not often discussed but will be the subject of investigation in this paper.

We find that for the response of labor supply, the size and response of firm profits play an essential role. Because prices are sticky, average price markups and profits respond countercyclically to monetary policy shocks. The countercyclical response of profits means that the representative agent experiences a countercyclical income effect, which directly causes a procyclical labor supply response. In addition, under a standard parameterization monopolistic profits make 45 \% of total output in steady state. When returned to a representative agent, this large non-labor income reduces the relative income effect of wages, which reduces the slope of the labor supply schedule.    

We illustrate this mechanism by means of simple modification of the textbook-model to allow for heterogeneity in the functional distribution of income. The modified model has two classes of agents: workers that only receive labor income and capitalists who own the firms. To make the exposition as clear as possible we first consider the case without financial trade between the two agents. In this model, under the standard assumption of King-Plosser-Rebelo (KPR) preferences, labor supply and output is unresponsive to monetary policy shocks. As a robustness exercise, we then allow for financial trade under the assumption of quadratic bond holing costs (to maintain stationarity). Even with small bond holdings costs, the output response is still close to non-existent.

Why does the worker-capitalist model not feature any response of output? Because workers are not given any profits, they do not experience the countercyclical income effect. In addition, because workers consume only labor income, KPR-preferences imply that the income and substitution effect of changes in the wage level cancel, and so labor supply is unresponsive to changes in the real wage. 

The implausible transmission mechanism through profits to employment and output follows from the assumption of frictionless labor markets. We argue that a model with nominal wage rigidities can serve as a better benchmark. In such environments, workers are off their long-run labor supply schedule and supply whatever is demanded at the going wage. The income and substitution effects therefore matter less. We show this in the context of the \citet{Erceg2000} model. In this framework, the standard and the modified worker-capitalist model produce very similar aggregate responses to a monetary policy shock. 

Our discussion is centered around monetary policy shocks. In the last section of the paper, we discuss another impulse-response in which the 3-equation model has been considered successful; the employment response to TFP shocks. Subject to a positive TFP shock, the 3-equation model generates a fall in employment, which is consistent with findings in the VAR evidence \citep{Gali1999,Gali2004,Francis2005,Basu2006}. We show that this impulse-response is also an artifact of profits being redistributed to the representative agent. In the worker-capitalist modification, employment does not move. The mechanism behind the fall in employment in the standard model is the large and procyclical response of profits, which forms a direct income effect sizable enough to cause a depression of labor supply even though wages increase. 
 
Our paper relates directly to two strands in the literature. Most directly, since we highlight a problematic property of the 3-equation model, our analysis relates to a number of studies that uses this model for analysis of the business cycle and business cycle policy. To name a few, \citet{Lorenzoni2009} studies the effect of news shocks. \citet{Christiano2011} analyzes government spending multipliers. \citet{Werning2012} characterize optimal fiscal and monetary policy at the zero lower bound. 

Secondly, our analysis connects the recent literature that has analyzed the effect of household income and wealth heterogeneity in New-Keynesian environments, since our worker-capitalist model can be seen as highly stylized model of the same type of heterogeneity. \citet{Auclert2015} studies amplification of monetary policy through redistribution and heterogeneity in marginal propensities to consume. \citet{McKay2015} studies how the precautionary savings motive affects the responsiveness to forward-guiding monetary policy. \citet{McKay2013} and \citet{Ravn2012a} studies richer model to uncover the effect of job uncertainty and automatic stabilizers. 