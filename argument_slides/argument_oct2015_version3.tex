\documentclass{beamer}
\usepackage[authoryear]{natbib}
\usepackage[latin1]{inputenc}
\usepackage[T1]{fontenc}
\usepackage{graphicx, amsmath, tabularx}
\usetheme{default}
\setbeamercovered{transparent}
\usepackage{soul}
\usepackage{natbib}
\usepackage[makeroom]{cancel}
\bibliographystyle{econometrica}



\newcommand{\bit}{\begin{itemize}}
\newcommand{\eit}{\end{itemize}}

\newcommand{\lb}{\label}
\newcommand{\re}{\eqref}

\title{The New Keynesian Transmission Mechanism}
\author{Tobias Broer, Niels-Jakob Harbo Hansen, Per Krusell and \\ Erik {\"O}berg}
\institute{Institute for International Economic Studies}

\begin{document}

\begin{frame}
\maketitle
\end{frame}

\section{Introduction}
\subsection{yyy}

\begin{frame}{Intro 1: Motivation I}

\bit
	\item The 3-equation model is widely used because it is a minimal model that can capture a key IR:
		\bit
			\item Output to monetary shocks
		\eit
	\item It is also a minimal model that captures a widely recognized intuition: Demand-driven fluctuations through intertemporal substitution
	\item As a consequence, it is the main vehicle for investigating
		\bit
			\item qualitative feature of various policy tools, such as forward guidance (e.g. Werning 2012) and fiscal multipliers (e.g. Woodford, 2011)
			\item amplification and mitigating mechanisms, such as financial acceleration (BGG, 1999) and the effect of credit constraints (McKay et al, 2015)
		\eit
\eit


\end{frame}

\begin{frame}{Intro 2: Motivation II}

\bit
	\item However, The 3-equation does not produce movements in output solely from movements in aggregate demand, but also in any other GE model, movements in supply
	\item In particular, the demand for output has to square with labor supply of the representative agent
	\item This is an element that is typically not stressed when discussing the model  
\eit


\end{frame}

\begin{frame}{Intro 3: Claim}

\bit
	\item We take the supply side of the model seriously and ask what are the driving factors in the determination of labor supply
	\item Our result: Labor supply moves in tandem with consumption demand because firm profits
		\begin{enumerate}
		\item are given to the \emph{representative agent}
		\item are \emph{large} and responds \emph{countercyclically}
		\end{enumerate}
	\item Mechanisms
		\bit
			\item Large profits: Reduces income effect of wages
			\item Countercyclical profits: Direct income effect
		\eit
	\item That profits are large and countercyclical is well-known, but to our knowledge not well known that these features are essential for the model performance
\eit


\end{frame}


\begin{frame}{Intro 4: What we do}

\bit
	\item To make the argument as clean as possible we compare the 3-equation model under BGP-preferences to a model where firms profits are given to capitalists outside the model
	\item In the WC model, there is no effect from monetary policy on output
	\item This is not an artifact of our unrealistic thought experiment
		\bit
			\item We construct a more realistic model with financial trade between workers and capitalist
			\item Without trading costs, the model generates an increase in debt-to-income of approximately 600 \% to a 25 basis point shock
			\item If we constraint the debt response a little by increasing the trading costs, the output response is eliminated
		\eit	
\eit


\end{frame}



\begin{frame}{Intro 5: Generality?}

\bit
	\item The profit channel arises due to the assumption of frictionless labor supply
	\item It should not extend into environments where workers are constrained to supply whatever is demanded, e.g. the Erceg (2000) model
	\item In fact, we show that there is little or no difference between the representative agent and the WC model in the Erceg framework
\eit


\end{frame}

\begin{frame}{Intro 6: TFP}

\bit
	\item We also discuss the IRs to TFP shocks, another area in which the 3-equation has been deemed successful 
	\item It can match the countercyclical response of hours (Gali, 1999)
	\item We show that this result is also an artifact of the response of profits
	\item Hours decline because the increase in profits is so large that workers want to work less even though wages rise
\eit


\end{frame}

\begin{frame}{Intro 7: Consequences}

\bit
	\item We highlight a problematic feature of the 3-equation model
	\item It does not mean that the intertemporal substitution and aggregate demand are dead ends
	\item It means that without additional features, the intertemporal substitution and aggregate demand channel is difficult to square with movements in labor supply
	\item I.e. the 3-equation model is this sense to minimal for discussion of monetary policy and TFP shocks
\eit


\end{frame}


\begin{frame}{New section: Models}

\bit
	\item Presentation of the similarity and difference between the standard and WC model
	\item Why this particular WC model?
		\bit
			\item The essential property is that only labor income is consumed by workers in equilibrium
			\item This could be achieved by making workers hand to mouth instead
			\item But due to profits being countercyclical, the Taylor rule has to be inverted
			\item Ergo, our WC model is the simplest way of removing profits while maintaining rest of the model constant 
		\eit
\eit


\end{frame}

\begin{frame}{IRs of the two models to monetary policy shocks}

\bit
	\item Model outcomes are identical beside the behavior of hours and output
	\item What is going on? 
\eit


\end{frame}


\begin{frame}{Explanation}

\bit
	\item Under BGP preferences, hours are determined by $\frac{D_t}{W_t}$
	\item Large profits in steady state reduces inomce effect of wages
	\item Countercylical profits becomes a direct income effect
\eit


\end{frame}

\begin{frame}{Allowing bond trade}

\bit
	\item To show that the results are not an artifact of the way we constrain demand in the WC model, we now allow for bond trade between workers and capitalists
	\item To close this nonstationary model, we add bond holding costs (Schmitt-Grohe, 200)
	\item With costs $\rightarrow$ 0, the bond trade model is very similar to the representative agent model, but with the added response that debt-to-income increase by 600 \% to a 25 basis point shock
	\item When we constrain this response to be sensical, the output response quickly dies
\eit


\end{frame}



\begin{frame}{Introducing rigid wages}

\bit
	\item The effect of profits goes through the determination of labor supply
	\item In models where employment is demand determined we should not expect the profit channel to be operating
	\item We show this by studying one such model, the Erceg et al (2000)
\eit


\end{frame}

\begin{frame}{IRs of the two models to monetary policy shocks}

\bit
	\item Model outcomes are identical 
	\item What is going on? 
\eit


\end{frame}


\begin{frame}{Explanation}

\bit
	\item Under wage rigidities, hours are determined by labor demand 
	\item Auxiliary result: Profits become procyclical and so capitalist contribution to demand becomes procyclical
\eit


\end{frame}

\begin{frame}{TFP shocks}

\bit
	\item We have shown that the effect of monetary policy shocks in the model without wage rigidities rely on the counterfactual profit channel
	\item We now show that it also account for another IR which have been deemed successful by many researchers: Countercyclical response of hours to TFP shocks
	\item Describe experiment
\eit


\end{frame}

\begin{frame}{Explanation}

\bit
	\item The profit response dominates the wage response
\eit


\end{frame}


\end{document}




\begin{frame}{The New Keynesian Framework}

\bit
	\item Central predictions
	\begin{enumerate}
		\item Stability around interest rate target
		\item Inflation respond positively to positive demand (monetary) shocks
		\item Inflation respond negatively to positive supply (tfp) shocks
		\item Output respond positively to positive demand (monetary) shocks
		\item Output responds positively while hours respond negatively to positive supply (tfp) shocks
	\end{enumerate}
	\item Strongest evidence has been found for point 4 and 5
	\item Standard interpretation: These results follow from ''demand-driven'' output fluctuations	
\eit
\end{frame}



\begin{frame}{Summary of argument}

\bit
	\item Empirical regularity: Output moves to shocks to monetary policy
	\item Interpretation: Demand-driven fluctuations through intertemporal substitution
	\item Minimal model designed to capture this intuition: The 3-equation NK model
	\item But output must not only follow the consumption path but also the factor income and labor supply path
	\item For things to click, profit income must be large and countercyclical
	\item Minimal model cannot convey the intertemporal substitution intuition without the cost of absurd movements in factor income
	\item What model can?
		\bit
			\item Rigid labor markets
		\eit
\eit


\end{frame}