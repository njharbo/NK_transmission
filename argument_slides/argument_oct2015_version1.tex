\documentclass{beamer}
\usepackage[authoryear]{natbib}
\usepackage[latin1]{inputenc}
\usepackage[T1]{fontenc}
\usepackage{graphicx, amsmath, tabularx}
\usetheme{default}
\setbeamercovered{transparent}
\usepackage{soul}
\usepackage{natbib}
\usepackage[makeroom]{cancel}
\bibliographystyle{econometrica}



\newcommand{\bit}{\begin{itemize}}
\newcommand{\eit}{\end{itemize}}

\newcommand{\lb}{\label}
\newcommand{\re}{\eqref}

\title{The New Keynesian Transmission Mechanism}
\author{Tobias Broer, Niels-Jakob Harbo Hansen, Per Krusell and \\ Erik {\"O}berg}
\institute{Institute for International Economic Studies}

\begin{document}

\begin{frame}
\maketitle
\end{frame}

\section{Introduction}
\subsection{yyy}

\begin{frame}{Two main questions for the introduction}

\bit
	\item Why is 3-equation model important?
	\bit
		\item Minimal model designed to capture the intuition of demand-driven fluctuations in output and inflation through intertemporal substitution
		\item Main vehicle for discussions centered around this intuition
			\bit
				\item Monetary policy (including forward guidance)
				\item Determinacy and stablization
			\eit
		\item Benchmark for adding other demand-channels: Most notably distribution of MPC
			\bit
				\item McKay and Reis (2014), Auclert (2015)
			\eit
		\item Accordingly, it is what we teach students
	\eit
	\item Why is the profit channel in the 3-equation labor market model important?
	\bit
		\item It tells us that when augmenting the minimal model with a labor market things get complicated
		\item With supply-determined labor, the model needs a countercyclical response of profits
		\item With demand-determined labor, the model cannot produce much movement in inflation
	\eit
	\item Hence, do we need other transmission channels?
		\bit
			\item Capital formation
		\eit
\eit


\end{frame}

\begin{frame}{Intro 1: Motivation}

\bit
	\item The New Keynesian 3-equation model is the minimal model designed to capture the intuition of demand-driven fluctuations in output and inflation through intertemporal substitution
	\item Main vehicle for discussions centered around this intuition, e.g. monetary policy
	\item Benchmark for adding other demand-channels, such as heterogeneity in MPC
	\item Teaching device
\eit


\end{frame}


\begin{frame}{Intro 2: Main result}

\bit
	\item Claim: The 3-equation model IR results are a consequence of profits being large and countercyclical
	\item We show this by comparing the IRs of the 3-equation model to a model where workers must not consume any profits in equilibrium: The WC model
	\item Channels:
		\bit
			\item Profits being large depresses relative income effect of wages
			\item Profits being countercyclical forms a countercyclical income effect in itself
		\eit
\eit


\end{frame}

\begin{frame}{Intro 3: Wage rigidities}

\bit
	\item When adding rigid wages to the model, the difference between the two models vanish
	\item I.e. profits do no longer play any crucial role for the determination of output
	\item The reason is that labor supply become demand-determined
\eit


\end{frame}

\begin{frame}{Intro 4: TFP result}

\bit
	\item The profits-labor supply channel also accounts for another successful IR: Countercyclical movement in hours to TFP shocks
	\item This result does not survive the WC pertubation
	\item As with monetary policy shocks, the difference goes away when adding rigid wages 
\eit


\end{frame}

\begin{frame}{Intro 5: Consequences}

\bit
	\item Without rigid labor markets, there is little hope that the intertemporal substitution channel will make sense
	\item Consequently, it is first order importance to uncover the empirical prevalence of wage rigidities
\eit


\end{frame}

\begin{frame}{Intro 6: Related literature}

\bit
	\item 
\eit


\end{frame}

\begin{frame}{Models}

\bit
	\item Presentation of the similarity and difference between the standard and WC model
	\item Why this particular WC model?
		\bit
			\item The essential property is that only labor income is consumed by workers in equilibrium
			\item This could be achieved by making workers hand to mouth instead
			\item But due to profits being countercyclical, the Taylor rule has to be inverted
			\item Ergo, this model is the simplest way of removing profits while maintaining rest of the model constant 
		\eit
\eit


\end{frame}

\begin{frame}{IRs of the two models to monetary policy shocks}

\bit
	\item Model outcomes are identical beside the behavior of hours and output
	\item What is going on? 
\eit


\end{frame}


\begin{frame}{Explanation}

\bit
	\item Under BGP preferences, hours are determined by $\frac{D_t}{W_t}$
	\item Large profits in steady state reduces inomce effect of wages
	\item Countercylical profits becomes a direct income effect
\eit


\end{frame}


\begin{frame}{Introducing rigid wages}

\bit
	\item The effect of profits goes through the determination of labor supply
	\item Under rigid wages, employment becomes demand determined and we should not expect the profit channel to be operating there
	\item We introduce wage rigidities as in Erceg et al (2000)
\eit


\end{frame}

\begin{frame}{IRs of the two models to monetary policy shocks}

\bit
	\item Model outcomes are identical 
	\item What is going on? 
\eit


\end{frame}


\begin{frame}{Explanation}

\bit
	\item Under wage rigidities, hours are determined by labor demand 
	\item Auxiliary result: Profits become procyclical and so capitalist contribution to demand becomes procyclical
\eit


\end{frame}

\begin{frame}{TFP shocks}

\bit
	\item We have shown that the effect of monetary policy shocks in the model without wage rigidities rely on the counterfactual profit channel
	\item We now show that it also account for another IR which have been deemed successful by many researchers: Countercyclical response of hours to TFP shocks
	\item Describe experiment
\eit


\end{frame}

\begin{frame}{Explanation}

\bit
	\item The profit response dominates the wage response
\eit


\end{frame}

\begin{frame}{Conclusion}

\bit
	\item
\eit


\end{frame}


\end{document}




\begin{frame}{The New Keynesian Framework}

\bit
	\item Central predictions
	\begin{enumerate}
		\item Stability around interest rate target
		\item Inflation respond positively to positive demand (monetary) shocks
		\item Inflation respond negatively to positive supply (tfp) shocks
		\item Output respond positively to positive demand (monetary) shocks
		\item Output responds positively while hours respond negatively to positive supply (tfp) shocks
	\end{enumerate}
	\item Strongest evidence has been found for point 4 and 5
	\item Standard interpretation: These results follow from ''demand-driven'' output fluctuations	
\eit
\end{frame}



\begin{frame}{Summary of argument}

\bit
	\item Empirical regularity: Output moves to shocks to monetary policy
	\item Interpretation: Demand-driven fluctuations through intertemporal substitution
	\item Minimal model designed to capture this intuition: The 3-equation NK model
	\item But output must not only follow the consumption path but also the factor income and labor supply path
	\item For things to click, profit income must be large and countercyclical
	\item Minimal model cannot convey the intertemporal substitution intuition without the cost of absurd movements in factor income
	\item What model can?
		\bit
			\item Rigid labor markets
		\eit
\eit


\end{frame}