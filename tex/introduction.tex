\section{Introduction}
\lb{introduction}

Monetary policy in New Keynesian DSGE models is usually studied under the assumption of a representative agent. This is of course an unrealistic simplification, but the analysis can still be justified insofar that the model approximates the transmission mechanism of a more complex reality with heterogeneity and incomplete markets. 

In this paper, we investigate this presumption. For this purpose we extend an otherwise simple textbook version of the New Keynesian model to contain two classes of households: workers who earn wages, and capitalists who earn profits. We study the transmission of monetary policy shocks in this framework under two different assumptions about the sources of nominal rigidities in the model. First, we let all rigidities stem from the goods market. Second, we also include frictions in the labor market. 

We show that the source of nominal rigidities matters greatly for the monetary transmission mechanism in this framework. In the model with only price rigidities, there is no effect on output from monetary policy. This result is in sharp contrast to the corresponding representative agent model. In the model with frictional labor market, the model behaves closely to the representative agent model.

What is the intuition behind these results? In these models, labor is the sole factor of production. With rigidities only in the goods market, workers respond to wages according to their long-run labor supply curve. With balance growth path preferences and without profit income, the income and substitution effect from changes in the wage level cancel out. Consequently, changes in the wage level will not be able to affect employment, and output becomes invariant to monetary policy. This does not mean that monetary policy is neutral. To the contrary, there are strong redistributional effects. As is well-known regarding this class of models, profits respond countercyclically to demand shocks, making capitalists poorer while workers richer in response to a surprise cut in the nominal interest rate.

In the model with rigid wage setting, most workers are constrained to supply whatever labor is demanded in the short run. The response of employment, and consequently output, is therefore directly determined by the response in consumption demand following the monetary policy shock. Under a standard calibration, the response of consumption demand is similar to the representative agent model, and therefore heterogeneity adds little to the aggregate response of output.   

Besides shedding light on how the transmission mechanism of monetary policy interacts with factor income heterogeneity, our results also highlight a previously unappreciated feature of the transmission mechanism in the representative agent model with rigidities solely in the goods market. This model often serves as the benchmark for business cycle and policy analysis in the New-Keynesian literature (see e.g. \citet{Lorenzoni2009}, \citet{Christiano2011} and \citet{Werning2012}). We show that the steady state size as well as the countercyclical response of profits play a key role for the employment and output response to monetary policy shocks in this environment. First, the larger the share of profits in steady state, the lower becomes the relative income effect of changes in the wage level on the representative agent's labor supply. The labor supply schedule becomes less steep and any given change in the wage level induces a larger movement in labor supply. Second, the countercyclical response of profits gives rise to an income effect which further amplifies the procyclical labor supply response. In a standard calibration of the model, the steady state share of profits is 45 percent and the response of profits is close to three times as large as total income, making the two effects each account for about half of total effect on employment.

Our analysis connects to the recent and growing literature analyzing the effects of monetary policy under incomplete markets in New-Keynesian environments. This literature has mainly focused on how incomplete markets may alter the demand effects in the model, whereas our focus is on supply factors. We view the two approaches as complementary. To mention a few contributions, \citet{Auclert2015} studies amplification of monetary policy through redistribution via unhedged interest rate exposure and nominal contracts. \citet{McKay2015} studies how incomplete markets affects the responsiveness of consumption demand to forward-guiding monetary policy. Kaplan et al. (2016) [Cite!] calibrate a state-of-the-art medium scale model and analyze how the presence of hand-to-mouth agents affects the transmission mechanism. Werning (2015) [Cite!] poses similar questions, but in a simpler, analytically solvable, model. [TBD: should we mention something about GHH-preferences here?]

TBD: This paper proceeds ....