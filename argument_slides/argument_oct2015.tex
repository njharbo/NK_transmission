\documentclass{beamer}
\usepackage[authoryear]{natbib}
\usepackage[latin1]{inputenc}
\usepackage[T1]{fontenc}
\usepackage{graphicx, amsmath, tabularx}
\usetheme{default}
\setbeamercovered{transparent}
\usepackage{soul}
\usepackage{natbib}
\usepackage[makeroom]{cancel}
\bibliographystyle{econometrica}



\newcommand{\bit}{\begin{itemize}}
\newcommand{\eit}{\end{itemize}}

\newcommand{\lb}{\label}
\newcommand{\re}{\eqref}

\title{The New Keynesian Transmission Mechanism}
\author{Tobias Broer, Niels-Jakob Harbo Hansen, Per Krusell and \\ Erik {\"O}berg}
\institute{Institute for International Economic Studies}

\begin{document}

\begin{frame}
\maketitle
\end{frame}

\section{Introduction}
\subsection{yyy}

\begin{frame}{Two main questions for the introduction}

\bit
	\item Why is 3-equation model important?
	\bit
		\item Minimal model designed to capture the intuition of demand-driven fluctuations in output and inflation through intertemporal substitution
		\item Main vehicle for discussions centered around this intuition
			\bit
				\item Monetary policy (including forward guidance)
				\item Determinacy and stablization
			\eit
		\item Benchmark for adding other demand-channels: Most notably distribution of MPC
			\bit
				\item McKay and Reis (2014), Auclert (2015)
			\eit
		\item Accordingly, it is what we teach students
	\eit
	\item Why is the profit channel in the 3-equation labor market model important?
	\bit
		\item It tells us that when augmenting the minimal model with a labor market things get complicated
		\item With supply-determined labor, the model needs a countercyclical response of profits
		\item With demand-determined labor, the model cannot produce much movement in inflation
	\eit
	\item Hence, do we need other transmission channels?
		\bit
			\item Capital formation
		\eit
\eit


\end{frame}

\begin{frame}{Intro 1: Motivation}

\bit
	\item The New Keynesian 3-equation model is the minimal model designed to capture the intuition of demand-driven fluctuations in output and inflation through intertemporal substitution
	\item Main vehicle for discussions centered around this intuition, e.g. monetary policy
	\item Benchmark for adding other demand-channels, such as heterogeneity in MPC
	\item Teaching device
\eit


\end{frame}


\begin{frame}{Intro 2: Result}

\bit
	\item Claim: The 3-equation model IR results are a consequence of profits being large and countercyclical
	\item We show this by comparing the IRs of the 3-equation model to a model where workers must not consume any profits in equilibrium: The WC model
	\item Channels:
		\bit
			\item Profits being large depresses relative income effect of wages
			\item Profits being countercyclical forms a countercyclical income effect in itself
		\eit
	\item 
\eit


\end{frame}

\begin{frame}{Intro 3: Related literature}

\bit
	\item 
	\item 
\eit


\end{frame}

\begin{frame}{Models}

\bit
	\item Presentation of the similarity and difference between the standard and WC model
\eit


\end{frame}






\end{document}







\begin{frame}{The New Keynesian Framework}

\bit
	\item Central predictions
	\begin{enumerate}
		\item Stability around interest rate target
		\item Inflation respond positively to positive demand (monetary) shocks
		\item Inflation respond negatively to positive supply (tfp) shocks
		\item Output respond positively to positive demand (monetary) shocks
		\item Output responds positively while hours respond negatively to positive supply (tfp) shocks
	\end{enumerate}
	\item Strongest evidence has been found for point 4 and 5
	\item Standard interpretation: These results follow from ''demand-driven'' output fluctuations	
\eit
\end{frame}




	\item We study how the results of this model are affected by 2 types of labor markets
	\bit
		\item Competitive
		\item Rigid
	\eit
	\item We show that the competitive labor market model suffers from a severe non-robustness: 
		\bit
			\item Heterogeneity in claims to firm profits
		\eit
	\item The reason is that profit income is countercyclical in the first model, which forms a negative income effect to a positive demand shocks 










\begin{frame}{The New Keynesian Framework}

\bit
	\item From an RBC perspective: Lower interest rates <-> lower return to capital <-> depression of labor supply and output
\eit
\end{frame}




\begin{frame}{Summary of argument}

\bit
	\item Empirical regularity: Output moves to shocks to monetary policy
	\item Interpretation: Demand-driven fluctuations through intertemporal substitution
	\item Minimal model designed to capture this intuition: The 3-equation NK model
	\item But output must not only follow the consumption path but also the factor income and labor supply path
	\item For things to click, profit income must be large and countercyclical
	\item Minimal model cannot convey the intertemporal substitution intuition without the cost of absurd movements in factor income
	\item What model can?
		\bit
			\item Rigid labor markets
		\eit
\eit


\end{frame}