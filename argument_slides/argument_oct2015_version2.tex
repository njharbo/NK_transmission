\documentclass{beamer}
\usepackage[authoryear]{natbib}
\usepackage[latin1]{inputenc}
\usepackage[T1]{fontenc}
\usepackage{graphicx, amsmath, tabularx}
\usetheme{default}
\setbeamercovered{transparent}
\usepackage{soul}
\usepackage{natbib}
\usepackage[makeroom]{cancel}
\bibliographystyle{econometrica}



\newcommand{\bit}{\begin{itemize}}
\newcommand{\eit}{\end{itemize}}

\newcommand{\lb}{\label}
\newcommand{\re}{\eqref}

\title{The New Keynesian Transmission Mechanism}
\author{Tobias Broer, Niels-Jakob Harbo Hansen, Per Krusell and \\ Erik {\"O}berg}
\institute{Institute for International Economic Studies}

\begin{document}

\begin{frame}
\maketitle
\end{frame}

\section{Introduction}
\subsection{yyy}


\begin{frame}{Intro 1: Motivation}

\bit
	\item The New Keynesian 3-equation model is the minimal model designed to capture the intuition of demand-driven fluctuations in output and inflation through intertemporal substitution
	\item Main vehicle for discussions centered around this intuition, e.g. monetary policy
	\item Benchmark for adding other demand-channels, such as heterogeneity in MPC
	\item Teaching device
\eit


\end{frame}

\begin{frame}{Intro 2: What we do}

\bit
		\item We study how the results of the 3-equation model is affected by adding 2 types of labor markets
		\bit
			\item Competitive
			\item Rigid
		\eit
		\item We show that the competitive labor market model suffers from a severe non-robustness: 
			\bit
				\item Heterogeneity in claims to firm profits
			\eit
		\item The reason is that profit income is countercyclical, which forms a negative income effect to a positive demand shocks 
\eit


\end{frame}


\begin{frame}{Intro 3: Heterogeneity}

\bit
	\item To make the argument as clean as possible, we study a model with minimal heterogeneity: 
		\bit
			\item Capitalists who own production technology
			\item Laborers who own productive labor
			\item Access to bond market under reasonably calibrated bond holding costs
		\eit
	\item Comparing the WC model and the representative agent version of both labor market models reveals that
		\bit
			\item Competitive labor markets cannot generate any action in output to monetary shock when adding heterogeneity
			\item Rigid labor market model is unaffected by adding heterogeneity
		\eit
\eit


\end{frame}

\begin{frame}{Intro 4: Consequences}

\bit
	\item 
\eit


\end{frame}

\begin{frame}{Intro 5: Related literature}

\bit
	\item 
	\item 
\eit


\end{frame}

\begin{frame}{Models}

\bit
	\item Presentation of the similarity and difference between the standard and WC model
\eit


\end{frame}




\end{document}







\begin{frame}{The New Keynesian Framework}

\bit
	\item Central predictions
	\begin{enumerate}
		\item Stability around interest rate target
		\item Inflation respond positively to positive demand (monetary) shocks
		\item Inflation respond negatively to positive supply (tfp) shocks
		\item Output respond positively to positive demand (monetary) shocks
		\item Output responds positively while hours respond negatively to positive supply (tfp) shocks
	\end{enumerate}
	\item Strongest evidence has been found for point 4 and 5
	\item Standard interpretation: These results follow from ''demand-driven'' output fluctuations	
\eit
\end{frame}














\begin{frame}{The New Keynesian Framework}

\bit
	\item From an RBC perspective: Lower interest rates <-> lower return to capital <-> depression of labor supply and output
\eit
\end{frame}




\begin{frame}{Summary of argument}

\bit
	\item Empirical regularity: Output moves to shocks to monetary policy
	\item Interpretation: Demand-driven fluctuations through intertemporal substitution
	\item Minimal model designed to capture this intuition: The 3-equation NK model
	\item But output must not only follow the consumption path but also the factor income and labor supply path
	\item For things to click, profit income must be large and countercyclical
	\item Minimal model cannot convey the intertemporal substitution intuition without the cost of absurd movements in factor income
	\item What model can?
		\bit
			\item Rigid labor markets
		\eit
\eit


\end{frame}