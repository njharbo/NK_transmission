\section{Introduction}
\lb{introduction}

[NK under CM studied as approx.] New Keynesian models are often studied under the assumption of complete market [homogenous agents]. This assumption seems hard to square with the empirical evidence [Source?], but the relevance of the models can still be justified insofar to be they serve as useful approximations of a world with incomplete markets.

[We challenge this presumption] In this paper we challenge this presumption. Specifically, we construct a simple heterogenous agent pertubation of the 3-equation textbook New-Keynesian model \citep{Gali2009}. Here two classes of agents exist: workers supply labor and only receive labor market income, while capitalists receive firm profits and do not supply any labor. To make the argument clear, we first consider the case without financial trade between workers and capitalists. This perturbation of the text-book model is not capable of matching the VAR evidence on monetary and productivity shocks. Output is invariant to monetary policy shocks, and hours is invariant to productivity shocks.

[Intuition] The intuition behind this result lies on the model's supply side. This is often ignored in the exposition of New Keynesian models with complete markets, but lies at the heart of the transmission mechanism. Indeed, since the worker is operating on her intratemporal optimality condition, labor supply is governed by substitution and income effects. As the model exhibits balanced growth path preferences, these two will cancel out in the absence of a source of non-labor income. Consequently, wage changes will not be able to affect labor supply, why output becomes invariant to both monetary and productivity shocks.

[Highlights counter-factual in the textbook model] Our result also high-lights a counter-factual, but unappreciated, feature of the transmission mechanism of the New-Keynesian textbook model. Usually, the demand side of the transmission mechanism in this model is emphasized.\footnote{The standard intuition is that subject to a surprise increase in the nominal interest rate, the real interest rate increase due to sticky prices, and aggregate demand falls due to intertemporal substitution. Since, in the absence of capital, consumption equals output in equilibrium, output falls.} However, as in any general equilibrium model, the movement in output must not only be consistent with consumption demand, but also with the supply of factor inputs, which in the textbook model solely consist of labor. To make labor supply move in response to a monetary shock, the size and distribution of profits play a key role. First, the mere existence of profits in the budget set of the representative agent decreases the relative income effect, and thus allows wages to affect labor supply even in the presence of balance growth path preferences. This effect is sizable since profits make up 45 \% of steady state output under a standard calibration of the textbook model. Second, owing to sticky prices average markups and profits respond countercyclically to monetary policy. This gives rise to a countercyclical income effect on labor supply, which helps to increase labor supply in response to a decrease in the nominal interest rate and \emph{vice versa}.

[Generality of argument.] Our argument also has relevance in less stylized New-Keynesian setups. Allowing for bond-trade between the two classes of agents still leaves output invariant to both monetary and productivity shocks [?]. Moreover, when augmenting the model with standard degree of wage stickyness the output response in the heterogenous agent pertubation is still XX \% smaller than under the complete market assumption. 

[Literature on NK models with agent heterogeneity] Our analysis connects to the recent literature analyzing the effect of household income and wealth heterogeneity in New-Keynesian environments, since our worker-capitalist model can be seen as highly stylized model of the same type of heterogeneity. \citet{Auclert2015} studies amplification of monetary policy through redistribution and heterogeneity in marginal propensities to consume. \citet{McKay2015} studies how the precautionary savings motive affects the responsiveness to forward-guiding monetary policy. \citet{McKay2013} and \citet{Ravn2012a} studies richer model to uncover the effect of job uncertainty and automatic stabilizers. 

[Literature using the text-book model]
Secondly, the paper relates to the literature using the 3-equation model for analysis of the business cycle and business cycle policy. To name a few, \citet{Lorenzoni2009} studies the effect of news shocks. \citet{Christiano2011} analyzes government spending multipliers. \citet{Werning2012} characterize optimal fiscal and monetary policy at the zero lower bound. 

[Organization.] The remainder of the paper is organized as follows. In Section XX we set up our heterogeneous agent pertubation of the standard 3-equation model and relate it to the textbook model. In Section XX we use these two models to analyze monetary and TFP shocks. Section XX discusses how our results extend to a heterogenous agent model with financial trade, while Section XX discusses how the existence of wage stickyness affect our results. Section XX concludes. 